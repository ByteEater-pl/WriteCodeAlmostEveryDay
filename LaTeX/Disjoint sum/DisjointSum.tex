\documentclass{article}
\usepackage{amsthm}
\theoremstyle{definition}
\newtheorem{definition}{Definition}
\newtheorem*{theorem}{Theorem}
\usepackage{mathtools}
\newcommand{\A}{\mathcal A}
\newcommand{\B}{\mathcal B}
\begin{document}
\begin{definition}\label{du1}
A \textbf{disjoint union} of sets $\A$ and $\B$ is the set:
\[
\A \uplus \B \coloneqq
\A \times \{0\} \cup \B \times \{1\}
\]
\end{definition}
\begin{definition}\label{du2}
A \textbf{disjoint union} of different sets $\A$ and $\B$ is the set:
\[
\A \uplus \B \coloneqq
\A \times \{\A\} \cup \B \times \{\B\}
\]
\end{definition}
\textbf{Definition~\ref{du1}} generalizes to sequences (including transfinite ones) of sets and indexed families, whereas \textbf{Definition~\ref{du2}} to families (but not multisets) of sets.
\begin{theorem}\[
\# \A \uplus \B =
\# \A + \# \B =
\# \A \cup \B + \# \A \cap \B
\]\end{theorem}
\end{document}
